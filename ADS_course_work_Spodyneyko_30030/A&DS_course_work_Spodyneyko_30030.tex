\documentclass[a4paper, 12pt] {article}

% Поддержка русского языка
\usepackage[T2A] {fontenc}              % Кодировка
\usepackage[utf8] {inputenc}            % Кодировка исходного текста
\usepackage[english, russian] {babel}   % Локализация и переносы

\usepackage{indentfirst}                % Делать отступ после разделов (section)
\usepackage{cmap}                       % Поиск и копирование в PDF
\usepackage{tikz}

\usepackage{indentfirst}	% Красная строка для первого параграфа

\usepackage{xcolor}


\usepackage{listings}
\lstset{
  language=Fortran,
  basicstyle=\ttfamily\normalsize,
  showstringspaces=false,
%  breaklines=true,
  tabsize=3,
  numbers=left,
  numbersep=4pt,
  numberstyle=\small\color{gray},
  numberfirstline=true,
  keywordstyle=\color{blue},
  commentstyle=\color{red},
  morecomment=[l]{!\ }
}

\usepackage{geometry}                   % Поля
   \geometry{left=20mm, right=15mm, top=20mm, bottom=20mm}

\title{Отчёт по Курсовой работе}

\begin{document}

%------------------------------Титульный_лист------------------------------%

\thispagestyle{empty}                   % Отключаем колонтитулы

\begin{center}
   Санкт-Петербургский политехнический университет Петра Великого\\
   Институт компьютерных наук и кибербезопасности\\
   Высшая школа программной инженерии
\end{center}

\vspace{20ex}                % Задаём размер вертикального промежутка в явном виде

\begin{center}
   \begin{LARGE}
     {\bfseries{\scshape курсовая работа}}
   \end{LARGE}

   \vspace{3ex}

   \begin{large}
     {\bfseries Разработка структур данных\\}
   по дисциплине: «Алгоритмы и Структуры Данных»
   \end{large}
\end{center}

\vspace{30ex}

\noindent Выполнил\\
студент гр. 30030/2x \hfill \begin{minipage}{0.6\textwidth}\hfill
  В.Ю.Сподынейко\end{minipage}

\vspace{3ex}

\noindent Руководитель\\
ст. преп. ВШПИ ИКНК\hfill \begin{minipage} {0.6\textwidth}\hfill
  С.А.Фёдоров\end{minipage}

\vfill

\begin{center}
   Санкт-Петербург\\
   2023
\end{center}

%-----------------------------Оглавление----------------------------------%
\newpage
\tableofcontents
%\contentsname{Оглавление}

%-----------------------------Задание_на_работу---------------------------%
\newpage

\begin{center}
  \section*{Задание\\ {\large Вариант 18}}
\end{center}

\begin{tabbing}
  \hspace{8em}\= \hspace{4em}\= \hspace{4em}\= \kill

  \textbf{Дан список группы в виде:}\\
  ФАМИЛИЯ     \> И. О.       \> ПОЛ         \> ГОД РОЖДЕНИЯ\\  
  15 симв.    \> 5 симв.     \> 1 симв.     \> 1 симв.\\
  \\
  \textbf{Пример входного файла:}\\
  \hspace{8em}\= \hspace{3.5em}\= \hspace{2em}\= \kill
  Иванов      \> И. Л.         \> М           \> 1985\\
  Петрова     \> Д. О.         \> Ж           \> 1983\\
\end{tabbing}
Сформировать отсортированные по убыванию возраста списки мужчин и
женщин. Использовать для сортировки метод “выбором”.\\
\begin{tabbing}
  \hspace{8em}\= \hspace{3.5em}\= \hspace{2em}\= \kill
  \textbf{Пример выходного файла:}\\
  Мужчины:\\
  Иванов         \> И. Л.      \> М           \> 1985\\
  \\
  Женщины:\\
  Петрова        \> Д. О.      \> Ж           \> 1983\\
\end{tabbing}

%-----------------------------Введение------------------------------------%
%\newpage
\begin{center}
  \section*{Введение}
\end{center}

Цель работы -- Выбор структуры данных для решения поставленной задачи
на современных микроархитектурах.
\subsection*{Задачи:}
\begin{enumerate}
  \item Реализовать задание с использованием массивов строк.
  \item Реализовать задание с использованием массивов символов.
  \item Реализовать задание с использованием массивов структур.
  \item Реализовать задание с использованием структур массивов.
  \item Реализовать задание с использованием массивов структур или
    структур массивов (на выбор) и с использованием хвостовой рекурсии при обработке данных.
  \item Реализовать задание с использованием динамического списка.
  \item Провести анализ на регулярный доступ к памяти.
  \item Провести анализ на векторизацию кода.
  \item Провести сравнительный анализ реализаций.
\end{enumerate}

%------------------------------Основная_часть----------------------------------------%
%------------------------------Глава_1-----------------------------------------------%
\newpage
\begin{center}
  \section{Реализация и анализ применения различных структур данных}
\end{center}

  Исходный сортируемый список состоит из 259\,000 объектов.\medskip
%------------------------------------------------------------------------------------%
%  * приводите объявление структуры данных (для конкретной вариации)
%  * приводите основные операторы обработки данных
%  * указываете где регулярный доступ, а где нет
%  * указываете где код был векторизован, а где потенциально векторизуем (например, тре-
%    буется выравнивание данных)
% ** приводите обоснование выбора назначения индексов в массиве символов
%    (Names(LEN, AMOUNT) или Names(AMOUNT, LEN))
%------------------------------------------------------------------------------------%
%----------------------------1_Массив_строк------------------------------------------%
\subsection{Массив строк}

  Данные в памяти сплошные. Регулярный доступ к памяти осуществляется при разбиении списка по
полу\footnote{см. строки 2-12}. При сортировке\footnote{см. строки
15-43} доступ к памяти - нерегулярный.
  
  Код векторизован при разбиении списка по полу\footnote{см. строки
  2-12} и при перестановке элементов\footnote{см. строки 24-26} (с помощью векторного индекса).
На других участках код не векторизуется или векторизация затруднена, так как имеются зависимости
"чтение после записи" и условные ветвления.

  Время обработки данных: 341.915с\bigskip

{\bfseries{\large Объявление струкутуры данных:}}
\lstinputlisting[language=Fortran]{source_code/01_strings_array_declaration.f90} 

\newpage
{\bfseries{\large Основные операторы обработки данных:}}
\lstinputlisting[language=Fortran]{source_code/01_strings_array_processing.f90} 

%----------------------------2_Массив_символов---------------------------------------%
\newpage
\subsection{Массив символов}

  Данные в памяти сплошные. Регулярный доступ к памяти осуществляется
при разбиении списка по полу\footnote{см. строки 3-17} и при
обращении к массивам символов во время сортировки\footnote{см. строки 35-38}.

  Код векторизован на строках 3-8, 14-15, 37 и при перестановке двух
элементов массива данных\footnote{см. строки 52-60}. На других
участках код не векторизуется так как имеются зависимости
"чтение после записи" и условные ветвления.\bigskip

  Хранение исходных массивов символов осуществляется по строкам.\\
\textit{Исходные массивы:}\\
\textbf{Surnames}(STUD\_AMOUNT, SURNAME\_LEN) и \textbf{Initials}(STUD\_AMOUNT, SURNAME\_LEN)\\
\textit{Расположение в памяти:}\\
\textbf{Surnames}(SURNAME\_LEN, STUD\_AMOUNT) и \textbf{Initials}(SURNAME\_LEN, STUD\_AMOUNT)\smallskip

  Это позволяет обеспечивать регулярный доступ к памяти при сравнении
двух строк исходного массива - они располагаются в памяти непрерывно
(сплошные данные). 
Обход массивов в памяти осуществляется по столбцам: Surnames(:, j) и
Initials(:, j). 

  Время обработки данных: 144.414с\bigskip\\

{\bfseries{\large Объявление струкутуры данных:}}
\lstinputlisting[language=Fortran]{source_code/02_symbols_array_declaration.f90} 

\newpage
{\bfseries{\large Основные операторы обработки данных:}}
\lstinputlisting[language=Fortran]{source_code/02_symbols_array_processing.f90} 

%----------------------------3_Массив_структур---------------------------------------%
\newpage
\subsection{Массив структур}

  Данные в памяти не сплошные. Доступ к памяти нерегулярный.

  Код векторизован при разбиении списка по полу\footnote{см. строки
  2-3} и при перестановке двух элементов списка\footnote{см. строку
  13}. На других участках код не векторизуется так как имеются
  зависимости "чтение после записи" и условные ветвления.

  Время обработки данных: 74.358с\bigskip\\

{\bfseries{\large Объявление струкутуры данных:}}
\lstinputlisting[language=Fortran]{source_code/03_array_of_structures_declaration.f90} 

\newpage
{\bfseries{\large Основные операторы обработки данных:}}
\lstinputlisting[language=Fortran]{source_code/03_array_of_structures_processing.f90} 

%----------------------------4_Структура_массивов------------------------------------%
\newpage
\subsection{Структура массивов}

  Данные в памяти сплошные. Регулярный доступ к памяти осуществляется при разбиении списка по
полу\footnote{см. строки 2-8}. При сортировке\footnote{см. строки
11-16 и 26-35} доступ к памяти - нерегулярный.

  Код векторизован при разбиении списка по полу\footnote{см. строки
2-8} и при перестановке двух элементов массива с помощью векторного
индекса\footnote{см. строки 19-22}

  Время обработки данных: 61.151с\bigskip\\
  
{\bfseries{\large Объявление струкутуры данных:}}
\lstinputlisting[language=Fortran]{source_code/04_structure_of_arrays_declaration.f90} 

\newpage
{\bfseries{\large Основные операторы обработки данных:}}
\lstinputlisting[language=Fortran]{source_code/04_structure_of_arrays_processing.f90}

%------------------5_Массив_структур_с_использованием_хвостовой_рекурсии-------------%
\newpage
\subsection{Массив структур с использованием хвостовой рекурсии при
  обработке данных}

  Данные в памяти не сплошные. Доступ к памяти нерегулярный.

  Код векторизован при разбиении списка по полу\footnote{см. строки 2-3} и при перестановке двух
элементов массива структур\footnote{см. строку 8}.

  Время обработки данных: 572.333с\bigskip\\

{\bfseries{\large Объявление струкутуры данных:}}
\lstinputlisting[language=Fortran]{source_code/05_array_of_structures_recursive_declaration.f90}

\newpage
{\bfseries{\large Основные операторы обработки данных:}}
\lstinputlisting[language=Fortran]{source_code/05_array_of_structures_recursive_processing.f90}

%---------------6_Структура_массивов_с_использованием_хвостовой_рекурсии-------------%
\newpage
\subsection{Структура массивов с использованием хвостовой рекурсии при
  обработке данных}

  Данные в памяти сплошные. Регулярный доступ к памяти осуществляется при разбиении списка по
полу\footnote{см. строки 2-8}. При сортировке\footnote{см. строки
11-19 и 22-41} доступ к памяти - нерегулярный.

  Код векторизован при разбиении списка по полу\footnote{см. строки
2-8} и при перестановке двух элементов массива с помощью векторного
индекса\footnote{см. строки 13-16}

  Время обработки данных: 183.642с\bigskip\\

{\bfseries{\large Объявление струкутуры данных:}}
\lstinputlisting[language=Fortran]{source_code/06_structure_of_arrays_recursive_declaration.f90}

\newpage
{\bfseries{\large Основные операторы обработки данных:}}
\lstinputlisting[language=Fortran]{source_code/06_structure_of_arrays_recursive_processing.f90}

%------------------7_Динамический_однонаправленный_список----------------------------%
\newpage
\subsection{Динамический однонаправленный список}

  Данные в памяти не сплошные. Доступ к памяти нерегулярный.

  Код не векторизуется.

  Время обработки данных: 161.108с\bigskip\\

{\bfseries{\large Объявление струкутуры данных:}}
\lstinputlisting[language=Fortran]{source_code/07_dynamic_list_declaration.f90}

\newpage
{\bfseries{\large Основные операторы обработки данных:}}
\lstinputlisting[language=Fortran]{source_code/07_dynamic_list_processing.f90}

%------------------------------Глава_2-----------------------------------------------%
\newpage
\begin{center}
  \section{Сравнение реализаций}
\end{center}

  По результатам выполнения задачи составлена сводная таблица, оценивающая
\textbf{участки кода по обработке данных}, где:\smallskip

\begin{itemize}
  \item[-] Сложность участка оценивалась как число строк кода.
  \item[-] Производительность участка кода - обратное время выполнения этого
    участка кода.
  \item[-] Эффективность участка кода - отношение производительности участка
    кода к его сложности.
\end{itemize}

  При компиляции выбран уровень оптимизации \textbf{-O3}\medskip

  Для решения данной задачи на современных микроархитектурах
наиболее эффективно использовать \textbf{структуру массивов}, так как время
обработки данных и эффективность участка кода по обработке данных у
этой реализации - наилучшие. 

  Если планируется масшатибрование кода в коммерческих целях, то следует рассмотреть
использование \textbf{массива структур}: сложность кода при этом минимальна, а
производительность незначительно уступает более производительной
реализации\footnote{на 18\%} - структуре массивов.\smallskip\\
  
\begin{table}[h] %[!b]
  \begin{small}
    \begin{tabular}{|p{7em}|p{4em}|p{4em}|p{4em}|p{5em}|p{5em}|p{5em}|p{5em}|}
      \hline & Массив строк & Массив символов & \textbf{Массив структур} &%
          Структура массивов & Массив структур с хвостовой рекурсией &%
          Структура массивов с хвостовой рекурсией &%
          Динамич. список\\ \hline
          
  Сплошные данные             & + & + & - & + & - & + & - \\ \hline
  Регулярный доступ           & + & + & - & + & - & + & - \\ \hline
  Векторизация \hspace{2em}   & + & + & + & + & + & + & - \\ \hline
  Потенциальная векторизация  & - & - & - & - & - & - & - \\ \hline
  Время работы кода           & 341.92c & 144.41c & 74.36с &
  61.15c & 572.33c & 183.64с & 161.11c \\ \hline
  Сложность \hspace{1em} кода & 67 & 88 & 40 & 53 & 45 & 61 & 66 \\ \hline
  Эффективность кода          & 43.7 & 78.7  & 336.2 & 308.5 & 38.8 &
  89.3 & 94.0 \\ \hline

  \end{tabular}
  \end{small}
  \caption{Сравнительная таблица реализаций структур данных}
\label{data_structures_table}
\end{table}

%------------------------------------------------------------------------------------%
\newpage
\section{Заключение}

  Задачи, решённые в курсовой работе:
\begin{itemize}
  \item[-] отсортированные списки сформированы
  \item[-] задача реализована с использованием структур данных,
    приведённых в \textbf{таблице \ref{data_structures_table}}.
  \item[-] проведён анализ на регулярный доступ к памяти
  \item[-] проведён анализ на векторизацию
  \item[-] проведён сравнительный анализ реализаций
\end{itemize}

  В ходе выполнения работы была неверно выбран \textit{"массив структур с хвостовой
рекурсией"}. Для исправления ошибки была предпринята реализация с
помощью \textit{"структуры массивов с хвостовой рекурсией"}.

  При разработке использовался процессор i5-1135G7, архитектура – x64
AMD64 (Intel 64), микроархитектура – Willow Cove, семейство
микроархитектуры – Tiger Lake.\\ \medskip

\textbf{Выводы:} 
\begin{enumerate}
  \item Цель по выбору структуры данных для формирования отсортированных
    списков студентов была достигнута, предпочтительная структура
    данных - \textbf{структура массивов}.
  \item При разработке необходимо своевременно использовать метрики эффективности и
    производительности кода чтобы избежать ошибок и финансовых издержек.
  \item Получен практический опыт использования различных структур
    данных.
  \item Получен опыт по приведению кода к единому стилю оформления. 
\end{enumerate}

%------------------------------------------------------------------------------------%
\end{document}   
